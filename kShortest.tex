%
\documentclass[10pt]{article} 

%% Use the option review to obtain double line spacing
%% \documentclass[preprint,review,12pt]{elsarticle}

%% Use the options 1p,twocolumn; 3p; 3p,twocolumn; 5p; or 5p,twocolumn
%% for a journal layout:
%% \documentclass[final,1p,times]{elsarticle}
%% \documentclass[final,1p,times,twocolumn]{elsarticle}
%% \documentclass[final,3p,times]{elsarticle}
%% \documentclass[final,3p,times,twocolumn]{elsarticle}
%% \documentclass[final,5p,times]{elsarticle}
%% \documentclass[final,5p,times,twocolumn]{elsarticle}

%% if you use PostScript figures in your article 
%% use the graphics package for simple commands
%% \usepackage{graphics}
%% or use the graphicx package for more complicated commands
%% \usepackage{graphicx}
%% or use the epsfig package if you prefer to use the old commands
%% \usepackage{epsfig}
\usepackage{textcomp}
\usepackage{xspace}
\usepackage{authblk}
%% The amssymb package provides various useful mathematical symbols
\usepackage{amsmath}
\usepackage{amssymb}
\usepackage{xspace}
\usepackage{verbatim}
\usepackage{longtable}
\usepackage[top=0.9in, bottom=0.9in, left=0.9in, right=0.9in]{geometry}
\linespread{1.05}         % Palatino needs more leading (space between lines)
%\usepackage[T1]{fontenc}
\usepackage[TS1,T1]{fontenc}
%\usepackage{mathpazo}
\usepackage[sc]{mathpazo}
%another font
%\usepackage{mathptmx}
%
%\usepackage{paralist}
%% The amsthm package provides extended theorem environments
%% \usepackage{amsthm}

%% The lineno packages adds line numbers. Start line numbering with
%% \begin{linenumbers}, end it with \end{linenumbers}. Or switch it on
%% for the whole article with \linenumbers after \end{frontmatter}.
%% \usepackage{lineno}

%% natbib.sty is loaded by default. However, natbib options can be
%% provided with \biboptions{...} command. Following options are
%% valid:

%%   round  -  round parentheses are used (default)
%%   square -  square brackets are used   [option]
%%   curly  -  curly braces are used      {option}
%%   angle  -  angle brackets are used    <option>
%%   semicolon  -  multiple citations separated by semi-colon
%%   colon  - same as semicolon, an earlier confusion
%%   comma  -  separated by comma
%%   numbers-  selects numerical citations
%%   super  -  numerical citations as superscripts
%%   sort   -  sorts multiple citations according to order in ref. list
%%   sort&compress   -  like sort, but also compresses numerical citations
%%   compress - compresses without sorting
%%
%% \biboptions{comma,round}

% \biboptions{}

\newcommand{\cpp}{C++\xspace}
\newcommand{\emcpp}{C\kern -0.0em\raise 0.4ex\hbox{\em++}\xspace}
\newcommand{\charm}{Charm++\xspace}
\newcommand{\chare}{\emph{chare}\xspace}
\newcommand{\chares}{\emph{chares}\xspace}
\newcommand{\EP}{\emph{entry method}\xspace}
\newcommand{\EPs}{\emph{entry methods}\xspace}

\newcommand{\sA}{Stage 1\xspace}
\newcommand{\sB}{Stage 2\xspace}
\newcommand{\slvrA}{\emph{Allocation Generator}\xspace}
\newcommand{\slvrB}{\emph{Scenario Evaluator}\xspace}
\newcommand{\slvrsB}{\emph{Scenario Evaluators}\xspace}
\newcommand{\sBmgr}{\emph{Work Allocator}\xspace}
\newcommand{\lp}{LP\xspace}
\newcommand{\cw}{\emph{Cut Window}\xspace}

\newcommand{\wh}{\widehat}
\newcommand{\wt}{\widetilde}
\newcommand{\thalf}{\textstyle \frac{1}{2}}
\newcommand{\nchoosek}[2]{\left(\begin{array}{c}#1\\#2\end{array}\right)}
\newcommand{\gap}{\vspace{0.1in}}
\newcommand{\epc}{\hspace{1pc}}
\newcommand{\sol}{\mbox{SOL}}
\newcommand{\lcp}{\mbox{LCP}}
\newcommand{\diag}{\mbox{Diag}}
\newtheorem{theorem}{Theorem}
\newtheorem{lemma}[theorem]{Lemma}
\newtheorem{lem}[theorem]{Lemma}
\newtheorem{proposition}[theorem]{Proposition}
\newtheorem{prop}[theorem]{Proposition}
\newtheorem{corollary}[theorem]{Corollary}
\newtheorem{cor}[theorem]{Corollary}
\newtheorem{remark}{Remark}
\newtheorem{rem}[theorem]{Remark}
\newtheorem{example}{Example}
\newtheorem{ex}{Example}
\newtheorem{definition}{Definition}
\newtheorem{defn}{Definition}
\newcommand{\argmax}{\operatornamewithlimits{\arg\max}}
\newcommand{\argmin}{\operatornamewithlimits{\arg\min}}
\hyphenation{stoch-astic}

\begin{document}


\title{Itinerary generation using a $k-$shortest paths approach}

\author[1]{Udatta Palekar}
\author[2]{Laxmikant Kale}
\author[2]{Michael Robson}
\author[2]{Akhil Langer}
\author[3]{Alok Tiwari\thanks{tiwari2@illinois.edu}}
\affil[1]{Department of Business Administration, University of Illinois at Urbana-Champaign}
\affil[2]{Department of Computer Science, University of Illinois at Urbana-Champaign}
\affil[3]{Department of Industrial \& Enterprise Systems Engineering, University of Illinois at Urbana-Champaign}


\renewcommand\Authands{, and }

%\author{University of Illinois at Urbana-Champaign, Urbana, Illinois 61801, U.S.A.}

\maketitle
\begin{abstract}
We describe a rational, systematic, and efficient approach to itinerary generation for dynamic mission replanning (DMR) problems using $k-$shortest paths. 
\end{abstract}

\section{Introduction}\label{sec:intro}
Consider the dynamic mission replanning (DMR) problems \cite{dmr-planning,dmr-execution}. The various variants of these problems consist of a large-scale stage-1 problem which may be viewed as a linear program or as an integer program and several large-scale stage-2 problems (at least one for each scenario) which too may be viewed as linear programs or as integer programs. The key variables in each subproblem are the \textit{itinerary} variables ($x_i$ or $x_{ii'}$ etc) which correspond to different itineraries. An itinerary itself is a path in (discretised) space-time that satisfies several `business rules' (such as respecting crew and aircraft `duty-cycles'). However, these business rules are often difficult to specify in advance, subject to change, or too complicated to incorporate directly into an optimizer. Therefore they are \textit{not} included in the DMR model constraints. In other words, the DMR models work with itineraries that have been filtered to satisfy business rules.

The present article proposes a unified, rational, algorithmic approach for generating itineraries suitable for the various DMR models. We noted above that an itinerary may be viewed as a path in discretised space-time. To incorporate duty-cycle rules, we just add another `dimension' to this (discrete) space-time and construct directed graphs on this set of grid points. The itineraries are then just directed paths in such graphs.

The rest of this article is organised as follows
\section{Itinerary generation problem}\label{sec:problem}
\subsection{Problem description}\label{subsec:desc}
Consider the stage-1 `planning' problem, considered as a linear program. The notation below is the same as before. Additionally, we define the duals associated with each constraint:

\begin{longtable}[h]{lll}
{\it Objective Function} &&\\
minimize$\displaystyle\ \sum_{i} \zeta_i x_i$ + $\displaystyle\sum_{i} \displaystyle\sum_{lt} \Gamma^f_l F_{ilt} x_i$ $\displaystyle +~\sum_{s=0}^S p_s Q_s$& & \\
\ \ where: $\displaystyle Q_s := \sum_{e_s=1}^{N_s}\left(\sum_{i} \zeta_{i}^{e_s} x^{e_s *}_{i}~+~\sum_{i} \sum_{lt} \Gamma^f_l F_{ilt} x^{e_s *}_{i} + 
\sum_{f} P_f \mu_f^* + \sum_{i} cost_{fi} z_{fi}^{e_s *}\right)$, defined for scenarios $s$& &\\
\ \ which occur with probabilities $p_s$ and have $N_s$ events $e_s$.& &\\
%& & \\
\end{longtable}
\begin{longtable}[h]{ll|l|l}
%& & \\
{\it Constraint} && {\it Dual} &\\
{\it Aircraft Constraints} &&&\\
$\displaystyle\sum_{i=1}^{I} \phi_{ijlt}x_i - \displaystyle\sum_{i=1}^{I} \psi_{ijl(t-R_{j}')}x_i + a_{jlt} - a_{jl(t-1)}$ & $= N_{jlt}$ &$\pi^A_{jlt}$ & $\forall j,l,t$\\
& & &\\
{\it Crew Constraints} &&& \\
$\displaystyle\sum_{i=1}^{I} \eta_{iklt}x_i - \displaystyle\sum_{i=1}^{I} \gamma_{ijl(t-R_k)}x_i + c_{klt} - c_{kl(t-1)}$ & $= C_{klt}$ &$\pi^C_{klt}$& $\forall k,l,t$ \\ 
& & &\\
{\it Fuel Constraints} &&& \\
$\displaystyle\sum_{i=1}^{I} F_{ilt}x_i + f_{lt} -f_{l(t-1)} $ & $ = G_{lt}$ &$\pi^F_{lt}$ & $\forall l,t$\\
%$\displaystyle\sum_{i=1}^{I} F_{ilt}x_i + f_{lt} -f_{l(t-1)} + \sum_{i, h \in rhop(i)} F_{ih}'rdep(l,i,h,t)v_{ihl}$ & $ = G_{lt}$ &$\pi^F_{lt}$ & $\forall l,t$\\
& & &\\
{\it Maximum on Ground} &&& \\
$\displaystyle\sum_{i=1}^{I}gs(i)\theta_{ilt}x_i - \displaystyle\sum_{i=1}^{I}gs(i)\omega_{ilt}x_i + m_{lt} - m_{l(t-1)}$ & $=M_{lt}$ &$\pi^M_{lt}$ & $\forall l,t$ %\\ 

%$\displaystyle\sum_{i,j}gs(i)J^{L}_{j}J_{ij}\theta_{ilt}x_i - \displaystyle\sum_{i,j}gs(i)J^{L}_{j}J_{ij}\omega_{ilt}x_i + m'_{lt} - m'_{l(t-1)}$ & $=M^L_{lt}$ &$\pi^{M,{\cal SB}}_{jlt}$ & $\forall l \in {\cal SB},t$ \\ 
%{\it Refueling constraints} &&& \\
%$\displaystyle\sum_{l \in rloc(i,h)} v_{ihl} - \nu_{ih}x_i $ & $ =0 $ &$\pi^R_{ih}$ & $\forall i, h \in rhop(i)$ \\ 
%$\displaystyle\sum_{i,h \in rhop(i)} rdep(l,i,h,t)v_{ihl} - \displaystyle\sum_{i,h \in rhop(i)} rarr(l,i,h,t)v_{ihl} + t_{lt} - t_{l(t-1)}$ & $ =N_{jlt} $ &$\pi^T_{jlt}$ & $\forall l,t, j = $ tanker-type\\ 
%& & \\
%{\it Alternative for Refueling constraints} && \\
%$\displaystyle\sum_{i=1}^{I} \tau_{ilt}x_i - \displaystyle\sum_{i=1}^{I} \rho_{ilt}x_i + t_{lt} - t_{l(t-1)}$ & $= N_{jlt}$ & $\forall l,t, j = $tanker-type\\
\end{longtable}

\subsection{Input data}\label{subsec:data}
We are given a \underline{basic} feasible solution of the above LP, a corresponding `basis', and the dual values associated with this solution. We write the constraints above as $Ey = r$ where 

\begin{equation*}
y=\Big((x_i),(a_{jlt}),(c_{klt}),(f_{lt}),(m_{lt})\Big)%,(m_{lt}^{'}),(t_{lt})\Big)
\end{equation*}

and 

\begin{equation*}
E = \Big[[E^{x}_{i}],[E^{a}_{jlt}],[E^{c}_{klt}],[E^{f}_{lt}],[E^{m}_{lt}]\Big]%,[E^{'m}_{lt}],[E^{t}_{lt}]\Big].
\end{equation*}

Let $y$ be a basic feasible solution with the basis $\mathcal{B} = \mathcal{B}_x \cup \mathcal{B}_a \cup \mathcal{B}_c \cup \mathcal{B}_f \cup \mathcal{B}_m$. %\cup \mathcal{B}_{m'} \cup \mathcal{B}_t$.

Then the basis matrix is:
\begin{align*}
E_{\mathcal{B}} = & \Big[E_{\alpha}\Big]_{\alpha \in \mathcal{B}}\\
				= & \Big[[E^{x}_{i}]_{i \in \mathcal{B}_x},[E^{a}_{jlt}]_{{(j,l,t)} \in \mathcal{B}_a},[E^{c}_{klt}]_{{(k,l,t)} \in \mathcal{B}_c},[E^{f}_{lt}]_{{(l,t)} \in \mathcal{B}_f},[E^{m}_{lt}]_{{(l,t)} \in \mathcal{B}_m}\Big]%,[E^{'m}_{lt}]_{{(l,t)} \in \mathcal{B}_{m'}},[E^{t}_{lt}]_{(l,t) \in \mathcal{B}_t}\Big].
\end{align*}

And the reduced-cost for a non-basic column $E^x_i$ is 
\begin{align*}
& \left(\zeta_i + \sum_{lt} \Gamma^f_l F_{ilt}\right) ~-~ & \left[\left(\zeta_{i'} + \sum_{lt} \Gamma^f_l F_{i'lt}\right)_{i' \in \mathcal{B}_x}, 0, \dots, 0\right] E_{\mathcal{B}}^{-1} E^{x}_{i}\\
= & \left(\zeta_i + \sum_{lt} \Gamma^f_l F_{ilt}\right) ~-~ & \pi \cdot E^{x}_{i}\\
= & \left(\zeta_i + \sum_{lt} \Gamma^f_l F_{ilt}\right) ~-~ & \left(\sum_{j,l,t} (\phi_{ijlt} - \psi_{ijl(t-R'_j)})\pi^{A}_{jlt} + \sum_{k,l,t} (\eta_{iklt} - \gamma_{ikl(t-R_k)})\pi^{C}_{klt}\right.\\
 & & \ \ + \left.\sum_{l,t} F_{ilt}\pi^{F}_{lt} + \textrm{gs}(i) \sum_{l,t} (\theta_{ilt} - \omega_{ilt})\pi^{M}_{lt}\right)\\%\\
% & & \ \ + \sum \pi^{M,{\cal SB}}_{lt} + \sum_{h \in \hops(i)}\nu_{ih}\pi^{R}_{ih} + \sum \pi^{T}_{jlt}
= & \left(\zeta_i + \sum_{lt} \Gamma^f_l F_{ilt}\right) ~-~ & \left(\sum_{j,l,t} \phi_{ijlt}\pi^{A}_{jlt} - \sum_{j,l,t}\psi_{ijlt}\pi^{A}_{jl(t+R'_j)} + \sum_{k,l,t} \eta_{iklt}\pi^{C}_{klt} - \sum_{k,l,t}\gamma_{iklt}\pi^{C}_{kl(t+R_k)}\right.\\
 & & \ \ + \left.\sum_{l,t} F_{ilt}\pi^{F}_{lt} + \textrm{gs}(i) \sum_{l,t} (\theta_{ilt} - \omega_{ilt})\pi^{M}_{lt}\right)%\\
\end{align*}

%We denote this basic feasible solution by $\Big((x_i^{b}),(a_{jlt}^{b}),(c_{klt}^{b}),(f_{lt}^{b}),(m_{lt}^{b}),(m_{lt}^{'b}),(t_{lt}^{b})\Big)$. 
\section{Graph description}\label{sec:graph}
\subsection{Nodes}\label{subsec:nodes}
The nodes in our graph consist of triples of the form $(l,t,d)$ where $l \in \mathcal{L}$, the set of locations\footnote{Each geographical location gets represented \underline{twice} in the set $\mathcal{L}$, once as an arrival location $l_A$ and once as a departure location $l_D$.}, $t \in \left\{ 0,\dots,\mathcal{N}_{\textrm{periods}}\right\}$, and $d \in \left\{0,\dots,\mathcal{M}_{\textrm{max duty periods}}\right\}$.
\subsection{Edges}\label{subsec:edges}
There are several different classes of edges in this graph. We enumerate these classes below:

\begin{enumerate}
	\item One edge that goes from an arrival node $(l_A,t,d)$ to a departure node $(l_D,t+\Delta,\max\{d+\Delta,\mathcal{M}\})$.
		\begin{itemize}
			\item Depending on the nature of the stop, the aircraft needs to remain on ground for a minimum length of time; this determines the value of $\Delta$.
		\end{itemize}
	\item Edges that go from a departure node $(l_D,t,d)$ to an arrival node $(l'_A,t+TT,\max\{d+TT,\mathcal{M}\})$\footnote{For some departure nodes, e.g., those with `high' values of $d$, there will be no such edge since there aren't enough crew duty hours available to complete the next leg.}.
		\begin{itemize}
			\item These edges represent the travel between two locations. $TT$ is the travel time between the locations.
			\item Special case: we allow for 'in-air delays' of up to $D_\textrm{max delay}$. In other words $TT$ can take the values $\{TT_{\textrm{min}},TT_{\textrm{min}}+1,\dots,TT_{\textrm{min}}+D_\textrm{max delay}\}$.
		\end{itemize}
	%\item Edges that go from a departure node to another departure node. These include:
	%	\begin{itemize}
	%		\item Ground delays: edges that go from $(l_D,t,d)$ to $(l_D,t+1,\max\{d+1,\mathcal{M}\})$.
	%		\item Rest overnight: edges that go from $(l_D,t,d)$ to $(l_D,t+\mathcal{T}_{\textrm{RON}},0)$.
	%		\item Crew swap: edges that go from $(l_D,t,d)$ to $(l_D,t+\mathcal{T}_{\textrm{CS}},0)$.
	%		\item Special case: The only out-going edge from a node of the form $(l_D,t,\mathcal{M})$ is to the node $(l_D,t+\mathcal{T}_{\textrm{RON}},0)$
	%	\end{itemize}
\end{enumerate}
\subsection{Node and edge weights}\label{subsec:weights}
The edge weights are determined as follows (note that the parameters $i,j,k$ are known):
\begin{itemize}
\item An edge that goes from an `arrival' node $({l_{0}}_A,t_0,d_0)$ to a `departure' node $({l_{0}}_D,t_1,d_1)$ carries a weight $\displaystyle \eta_{ijlt_1}\pi^{A}_{l_0t_1} - \psi_{ijlt_0}\pi^{A}_{l_0(t_0+R'_j)} + \eta_{iklt_1}\pi^{C}_{l_0t_1} - \gamma_{iklt_0}\pi^{C}_{l_0(t_0+R_k)} + F_{il_0t_1}\pi^{F}_{l_0t_1} + \textrm{gs}(i)\sum_{t_0 \le t \le t_1} \pi^{M}_{l_0t}$%\left(\theta_{il_0t_1}\pi^{M}_{l_0t_1} - \omega_{il_0t_0}\pi^{M}_{l_0t_0}\right) + F_{il_0t_1}\pi^{F}_{l_0t_1}$.%\sum_{t_0 \le t \le t_1} \textrm{gs}(i)\pi^{M}_{l_0t}$
\item An edge that goes from a `departure' node $({l_0}_D,t_0,d_0)$ to an `arrival' node $({l_1}_D,t_1,d_1)$ carries a weight $\Gamma^{f}_{l_0}F_{il_0t_0}\pi^{F}_{l_0t_0}$.
\item We don't yet know where to assign $\zeta_i$ as an edge-weight. 
\end{itemize}
%\[
%\textrm{Edge weight} ~=~ \sum_{t \ge (t_1-t_c)}\pi^{\textrm{crew}}_{kl_1t}\eta_{ikl_1t_1} + \sum_{t \ge (t_1-t_a)}\pi^{\textrm{acft}}_{jl_1t}\phi_{ijl_1t_1} + \pi^{\textrm{MoG}}_{l_1t_1}\theta_{il_1t_1}\textrm{gs}_j 
%- \sum_{t \ge (t_0+R_k)}\pi^{\textrm{crew}}_{kl_0t}\gamma_{ikl_0t_0} - \sum_{t \ge (t_0+R'_j)}\pi^{\textrm{acft}}_{jl_0t}\psi_{ijl_0t_0}
%\]
\newpage

where:
\begin{itemize}
\item The edge goes from a node of the form $(l_0,t_0,d_0)$ to a node of the form $(l_1,t_1,d_1)$.
\item $t_c$ is the time before take-off that a crew is ``picked up''.
\item $t_a$ is the time before take-off that an aircraft is picked up.
\item $R_k$ is the recuperation time (assumed to be fixed) for a crew that is dropped off.
\item $R'_j$ is the recuperation time (assumed to be fixed) for an aircraft that is dropped off.
\end{itemize}

%\begin{enumerate}
%\item Each `travel' edge carries a weight equal to the flying cost.
%\item Each edge that begins and ends at the same location carries zero weight.\footnote{This can be changed, depending on modeling choices. For example, an edge corresponding to a crew swap may carry a weight equal to crew costs.}
%\item Each node carries a weight obtained from the present values of the `duals' (e.g., an $(l,t)$ pair that is MoG constrained may carry a higher weight than one that has plenty of spare capacity).
%\end{enumerate}

\subsection{Weather disruptions}\label{subsec:weather}
If a weather disruption prevents landings or take-offs from certain $(l,t)$ pairs, we simply remove appropriate edges from the graphs. If this disruption limits the number of landings or take-offs, we may be able to discourage landings or take-offs by increasing the weights of such edges.
\section{Algorithm}\label{sec:algorithm}
With the above (directed) graph, we simply solve a standard $k-$shortest path problem where the `length' of a path is the sum of the weights of all the edges and nodes that it contains. This yields $k$ itineraries that satisfy crew rules.
\section{Numerical results}\label{sec:numerics}
\input{numerics}
\end{document}

