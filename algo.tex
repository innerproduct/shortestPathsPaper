%\documentclass{article}
%\usepackage{amsthm}
%\usepackage{verbatim}
%\newtheorem{thm}{Theorem}
%\begin{document}
%\long\def\/*#1*/{} %include multiline comments using \/* and */

%algorithm for the robust deviation shortest path
\paragraph{Algorithm 1 (robust deviation shortest path)}

Consider a (connected) directed acyclic graph $G$ with the vertex set $V = \{1, \dots, n\}$ and the edge set $E \subseteq V \times V$. We assume that the vertices are presented in a valid topological order. We also assume that there are a total of $m$ edges and $s$ scenarios (each scenario consists of a an assignment of weights or costs to edges). The weight of an edge $(i \rightarrow j)$ in scenario $s$ is denoted $w^s_{ij}$. Our goal is to find a path from $1$ to $n$ that minimizes the maximum deviation in cost across all scenarios.\cite{some_refs}

We now describe our algorithm:

\begin{description}
\item[Step 0] Solve the (ordinary) shortest path problems for each of the scenarios to obtain the quantities $l^s_j$, the cost of the \emph{shortest} path from $1$ to $j$ in scenario $s$.
\item[Step 1] For vertex $1$, define the quantities $c^s_1 = 0$, $\mu^s_1 = c^s_1 - l^s_1 = 0$, $\lambda_1 = \textrm{max}_s\left\{\mu^s_1\right\} = 0$. Here the superscript $s$ referes to the scenario and the subscript $1$ refers to the vertex. The quantities $c, ~\mu$ are respectively the \textit{cost} of the \emph{robust deviation} path to $1$,  and the cost deviation in choosing the robust path over the shortest path (these quantities depend on the scenario $s$). The quantity $\lambda$ is the \emph{maximum deviation} across all scenarios. 
\item[Step 2] For every vertex $j \in \left\{2,\dots, n\right\}$ let $P_j$ be the list of its immediate predecessors (presented in a topological order). We select a \emph{robust predecessor} to $j$ as follows:
	\begin{itemize}
		\item For $i \in R_j$, compute $\nu^s_i = c^s_i + w^s_{ij} - l^s_j$ and $\nu^{\textrm{max}}_i = \textrm{max}_s\left\{\nu^s_i\right\}$.
		\item Find $i_* = \textrm{argmax}_i\left\{\nu^{\textrm{max}_i}\right\}$.
	\end{itemize}
	Then $i_*$is the \emph{robust predecessor} of $j$.
\item[Step 3] Compute the quantities $c^s_j = c^s_{i_*} + w^s_{i_*j}$, $\mu^s_j = c^s_j - l^s_j$, and $\lambda_j = \textrm{max}_s\left\{\mu^s_j\right\}$.
\end{description}

\begin{thm}
The above algorithm produces a path that minimizes the maximum of the (cost) deviations from the shortest path in any scenario.
\end{thm}

\begin{proof}
We prove this result by induction on the vertices $j$ (which are considered in topological order). Clearly the basis of induction, i.e., $j=1$ is true: the trivial path from the vertex $1$ to itself is also the optimal robust deviation path. Now, as the induction hypothesis, let us assume that we have available the robust shortest paths from the vertex $1$ to any vertex $i$ where $i \le k$. We wish to show that, given this assumption, the above procedure produces a robust shortest path to the vertex $k+1$. This follows from the following lemma.
\end{proof}

\begin{lemma} If $(1,i_2,\dots,i_{k-1},i_k)$ is the sequence of vertices in a robust deviation shortest path from the vertex $1$ to the vertex $i_k$, then the truncated sequence $(1,i_1,\dots,i_{k-1})$ corresponds to a robust deviation shortest path from the vertex $1$ to the vertex $i_{k-1}$.
\end{lemma}
\begin{proof} We shall prove this result by contradiction. In other words, let us assume that there exists a path $\mathcal{P}_1$ from the vertex $1$ to the vertex $i_{k-1}$ that has a strictly smaller maximum deviation than the path $\mathcal{P}_0$ which corresponds to the above (truncated) sequence. Let $\mathcal{Q}_r$ be the continuation of the path $\mathcal{P}_r$ obtained by concatenating the edge $(i_{k-1} \rightarrow i_k)$ to it. Our assumptions above mean that $\mathcal{P}_1$ is the robust deviation shortest path from vertex $1$ to vertex $i_{k-1}$ whereas $\mathcal{Q}_0$ is the robust deviation shortest path from vertex $1$ to vertex $i_{k}$.

Let $\prescript{r}{}{c}^{s}_{i_{k-1}}$ be the cost of the path $\mathcal{P}_r$ in scenario $s$ and let ${l}^{s}_{i_{k-1}}$ be the cost of the shortest path from $1$ to $i_{k-1}$ in scenario $s$. Then, in scenario $s$, deviation of the (cost of the) path $\mathcal{P}_r$, is $\prescript{r}{}{\mu}^{s}_{i_{k-1}} = \prescript{r}{}{c}^{s}_{i_{k-1}} - {l}^{s}_{i_{k-1}}$. The optimality of $\mathcal{P}_1$ means that for some scenario $\tilde{s}$, we have:

\begin{equation*}
\prescript{1}{}{\mu}^{s}_{i_{k-1}} \le \prescript{1}{}{\lambda}_{i_{k-1}} < \prescript{0}{}{\lambda}_{i_{k-1}} = \prescript{0}{}{\mu}^{\tilde{s}}_{i_{k-1}} = \prescript{0}{}{c}^{\tilde{s}}_{i_{k-1}} - {l}^{\tilde{s}}_{i_{k-1}}
\end{equation*}

Note that the deviation from optimal for the path $\mathcal{Q}_r$ in scenario $s$ is

\begin{align*}
\prescript{r}{}{\mu}^{s}_{i_{k}} &= \left[\left({l}^{s}_{i_{k-1}} + \prescript{r}{}{\mu}^{s}_{i_{k-1}}\right) + w^s_{i_{k-1}i_{k}}\right] - {l}^{s}_{i_{k}}.\\
 &= \prescript{r}{}{\mu}^{s}_{i_{k-1}} + \left[\left({l}^{s}_{i_{k-1}} + w^s_{i_{k-1}i_{k}}\right) - {l}^{s}_{i_{k}}\right].
\end{align*}

\begin{assumption}
If we assume that the quantity $\left[\left({l}^{s}_{i_{k-1}} + w^s_{i_{k-1}i_{k}}\right) - {l}^{s}_{i_{k}}\right]$, which may be termed `edge deviation', is independent of the scenario $s$ (but may depend on the pair of vertices $i_{k-1}$ and $i_k$), we can complete the proof of the above lemma.
\end{assumption}

Under the above assumption, we note that there exists $\tilde{s}$ such that for all $s$, we have

\begin{align*}
\prescript{1}{}{\mu}^{s}_{i_{k}} &= \prescript{1}{}{\mu}^{s}_{i_{k-1}} + \left[\left({l}^{}_{i_{k-1}} + w^{}_{i_{k-1}i_{k}}\right) - {l}^{}_{i_{k}}\right]\\
&\le \prescript{1}{}{\lambda}^{}_{i_{k-1}}+ \left[\left({l}^{}_{i_{k-1}} + w^{}_{i_{k-1}i_{k}}\right) - {l}^{}_{i_{k}}\right]\\
&< \prescript{0}{}{\lambda}^{}_{i_{k-1}}+ \left[\left({l}^{}_{i_{k-1}} + w^{}_{i_{k-1}i_{k}}\right) - {l}^{}_{i_{k}}\right]\\
&= \prescript{0}{}{\mu}^{\tilde{s}}_{i_{k-1}}+ \left[\left({l}^{}_{i_{k-1}} + w^{}_{i_{k-1}i_{k}}\right) - {l}^{}_{i_{k}}\right]\\
&=\prescript{0}{}{\mu}^{\tilde{s}}_{i_{k}}
\end{align*}

Which contradicts the assumption that $\mathcal{Q}_0$ is the robust deviation shortest path from vertex $1$ to vertex $i_{k}$.

This completes the proof of the lemma.

\end{proof}



\begin{comment}
Therefore 

\begin{equation*}
\prescript{1}{}{\mu}^{s}_{i_{k}} \le \left[\left({l}^{s}_{i_{k-1}} + \prescript{1}{}\lambda_{i_{k-1}}\right) + w^s_{i_{k-1}i_{k}}\right] - {l}^{s}_{i_{k}} < \left[\left({l}^{s}_{i_{k-1}} + \prescript{0}{}\lambda_{i_{k-1}}\right) + w^s_{i_{k-1}i_{k}}\right] - {l}^{s}_{i_{k}}.
\end{equation*}
\end{comment}


\begin{comment}
Similarly, the optimality of $\mathcal{Q}_0$ means that 

\begin{equation*}
\prescript{0}{}{\mu}^{s}_{i_{k}} \le \prescript{0}{}{\lambda}_{i_{k}} \le \prescript{1}{}{\lambda}_{i_{k}} = \prescript{1}{}{\mu}^{\hat{s}}_{i_{k}} = \prescript{1}{}{c}^{\hat{s}}_{i_{k}} - {l}^{\hat{s}}_{i_{k-1}}
\end{equation*}
\end{comment}

\begin{comment}
First some notation: let the path produced by our algorithm be $\mathcal{P}^{\textrm{alg}} = (1 = v_1,\dots,v_k = n)$. Let $\mathcal{P} = (1 = w_1,\dots, w_l = n)$ be any other path. We shall show that the maximum deviation for the path $\mathcal{P}$ is no better than the maximum deviation for the path $\mathcal{P^{\textrm{alg}}}$. Suppose this is not so. Let $s_{\mathcal{P}}$ denote the scenario that maximises the deviation from optimal for the path $\mathcal{P}$. Furthermore let $r$ denote the minimal index such that $w_r \neq v_r$, and let $s$ denote the minimal index such that $w_{l-s} \neq v_{k-s}$. Clearly, $r \geq 2$ and $s \geq 1$.

We consider the path $\mathcal{P}^{\textrm{alg}}$ for the scenario $s_{\mathcal{P}}$. Our assumption that $\mathcal{P}$ has a smaller maximum deviation than $\mathcal{P^{\textrm{alg}}}$ implies that the cost of $\mathcal{P^{\textrm{alg}}}$ in the scenario $s_{\mathcal{P}}$ must be greater than the cost of $\mathcal{P}$. This implies that the subpath $(w_{r-1},\dots,w_{l-s+1})$ has a strictly lower cost than the subpath $(v_{r-1},\dots,v_{k-s+1})$ in the scenario $s_{\mathcal{P}}$ (these two subpaths share the start and end vertices). Without loss of generality\footnote{Requires a justification; not hard to write but TBD.}, we may assume that $v_f \neq w_g$ for all $r \le f \le k-s$ and $r \le g \le l-s$ -- so that the two subpaths do not share any vertices that aren't the start or the end. Consider the vertex $v_r$: the cost to reach it in the scenario $s_{\mathcal{P}}$
%\end{proof}
\end{comment}


\paragraph{Algorithm 2 ($k-$shortest robust deviation paths)}
\begin{description}
\item[Step 0:] For each scenario we construct a singe-source shortest path tree on the graph. This has a complexity of $\mathcal{O}(s m)$. At the end of this step we have an $(s+3)-$tuple as a label at each vertex consisting of its distances from the origin (vertex $1$) in each scenario, with two further labels, say, $-1$ and $+\infty$ added on to indicate the predecessor in the robust shortest path, and distance from the origin along the robust shortest path. At a vertex $i$ we denote its label as $(d_i^1,\dots,d_i^s,P_i,d_i^{\textrm{robust}},s^{*})$, where $s_i^{*}$ is the scenario for which the deviation is the greatest.

\item[Step 1:] For the source vertex define the robust path and calculate the cost of the robust path in each scenario\footnote{Using either the ARSP or MDSP definition. This distinction doesn't matter for the source vertex since all costs are zero.}.

\item[Step 2:] Process the remaining vertices in a valid topological order. For each vertex $v$, we determine a `robust path'\footnote{Or $k$-shortest paths} to it from the source as follows:
\begin{enumerate}
\item Let $\{e^v_1,\dots,e^v_l\}$ be all the incoming edges for the vertex $v$. Let us denote the corresponding predecessor vertices $\{u_1,\dots,u_l\}$ (we drop the dependence on $v$ from the notation). For each predecessor vertex $u_i$ we compute the following $S$ quantities, where $S$ is the number of scenarios: \begin{comment}$c^{\textrm{sp},s}_i + c^s_{u_iv}$ and\end{comment} 
$c^{\textrm{rp},s}_i + c^s_{u_iv}$. In the previous expression the superscript $\textrm{rp}$ indicates that the cost is for the `robust' path. The superscript $s$ indicates the scenario.
\item 

\end{enumerate}
\end{description}
\begin{comment}
\begin{description}
	\item[Step 0:] For each scenario we construct a singe-source shortest path tree on the graph. This has a complexity of $\mathcal{O}(s m)$. At the end of this step we have an $(s+3)-$tuple as a label at each vertex consisting of its distances from the origin (vertex $1$) in each scenario, with two further labels, say, $-1$ and $+\infty$ added on to indicate the predecessor in the robust shortest path, and distance from the origin along the robust shortest path. At a vertex $i$ we denote its label as $(d_i^1,\dots,d_i^s,P_i,d_i^{\textrm{robust}},s^{*})$, where $s_i^{*}$ is the scenario for which the deviation is the greatest.
	\item[Step 1:] We next go through the vertices in a topological order and, for each vertex $j$, we look at all its incoming edges $(i,j)$\footnote{It isn't necessary for the correctness of our algorithm but we go through the list of predecessors in the topological order as well.} and we update the last two entries of $j$'s label as follows:
		\begin{description}
			\item[Update $j$:] Let $d_{ij}^s$ be the weight on the edge $(i,j)$ in scenario $s$. We compute the biggest of the following quantities: $\{d_{ij}^s + d_i^s - d_j^s, d_{ij^{s_i^{*}}} + d_i^{\textrm{robust}} - d_j^{s_i^{*}}\}$, denoted $D_j^{\textrm{max}}$ which corresponds to scenario $s'$, and compare it to the current value of $d_j^{\textrm{robust}} - d_j^{s_j^{*}}$. If $D_j^{\textrm{max}}$ is smaller, the value $d_j^{\textrm{robust}}$ is replaced by $d_j^{s'} + D_j^{\textrm{max}}$ and the $P$ and $s$ values are updated correspondingly.
		\end{description}
		The complexity of this step is also $\mathcal{O}(s m)$.
	\item[Step 2:] Finally the path that minimizes the maximum deviation is produced by starting at vertex $n$ and looking at predecessors $P$ until we reach vertex $1$. The cost of this path in any scenario (and indeed the deviation from the minimum cost to reach vertex $n$ in any scenario) is trivial to calculate. The complexity of this step is no more than $\mathcal{O}(s m)$.
\end{description}
\end{comment}


%\end{document}