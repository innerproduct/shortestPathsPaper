%
\documentclass[8pt]{article} 

%% Use the option review to obtain double line spacing
%% \documentclass[preprint,review,12pt]{elsarticle}

%% Use the options 1p,twocolumn; 3p; 3p,twocolumn; 5p; or 5p,twocolumn
%% for a journal layout:
%% \documentclass[final,1p,times]{elsarticle}
%% \documentclass[final,1p,times,twocolumn]{elsarticle}
%% \documentclass[final,3p,times]{elsarticle}
%% \documentclass[final,3p,times,twocolumn]{elsarticle}
%% \documentclass[final,5p,times]{elsarticle}
%% \documentclass[final,5p,times,twocolumn]{elsarticle}

%% if you use PostScript figures in your article 
%% use the graphics package for simple commands
%% \usepackage{graphics}
%% or use the graphicx package for more complicated commands
%% \usepackage{graphicx}
%% or use the epsfig package if you prefer to use the old commands
%% \usepackage{epsfig}
\usepackage{textcomp}
\usepackage{xspace}
\usepackage{authblk}
%% The amssymb package provides various useful mathematical symbols
\usepackage{amsmath}
\usepackage{amssymb}
\usepackage{amsthm}

\usepackage{mathtools}
\usepackage{tensor}

\usepackage{xspace}
\usepackage{verbatim}
\usepackage{longtable}

\usepackage[a4paper, top=0.5in, bottom=0.5in, left=0.5in, right=0.5in]{geometry}
\linespread{1.05}         % Palatino needs more leading (space between lines)
%\usepackage[T1]{fontenc}
\usepackage[TS1,T1]{fontenc}
%\usepackage{mathpazo}
\usepackage[sc]{mathpazo}
%another font
%\usepackage{mathptmx}
%
%\usepackage{paralist}
%% The amsthm package provides extended theorem environments
%% \usepackage{amsthm}

%% The lineno packages adds line numbers. Start line numbering with
%% \begin{linenumbers}, end it with \end{linenumbers}. Or switch it on
%% for the whole article with \linenumbers after \end{frontmatter}.
%% \usepackage{lineno}

%% natbib.sty is loaded by default. However, natbib options can be
%% provided with \biboptions{...} command. Following options are
%% valid:

%%   round  -  round parentheses are used (default)
%%   square -  square brackets are used   [option]
%%   curly  -  curly braces are used      {option}
%%   angle  -  angle brackets are used    <option>
%%   semicolon  -  multiple citations separated by semi-colon
%%   colon  - same as semicolon, an earlier confusion
%%   comma  -  separated by comma
%%   numbers-  selects numerical citations
%%   super  -  numerical citations as superscripts
%%   sort   -  sorts multiple citations according to order in ref. list
%%   sort&compress   -  like sort, but also compresses numerical citations
%%   compress - compresses without sorting
%%
%% \biboptions{comma,round}

% \biboptions{}

\newcommand{\cpp}{C++\xspace}
\newcommand{\emcpp}{C\kern -0.0em\raise 0.4ex\hbox{\em++}\xspace}
\newcommand{\charm}{Charm++\xspace}
\newcommand{\chare}{\emph{chare}\xspace}
\newcommand{\chares}{\emph{chares}\xspace}
\newcommand{\EP}{\emph{entry method}\xspace}
\newcommand{\EPs}{\emph{entry methods}\xspace}

\newcommand{\sA}{Stage 1\xspace}
\newcommand{\sB}{Stage 2\xspace}
\newcommand{\slvrA}{\emph{Allocation Generator}\xspace}
\newcommand{\slvrB}{\emph{Scenario Evaluator}\xspace}
\newcommand{\slvrsB}{\emph{Scenario Evaluators}\xspace}
\newcommand{\sBmgr}{\emph{Work Allocator}\xspace}
\newcommand{\lp}{LP\xspace}
\newcommand{\cw}{\emph{Cut Window}\xspace}

\newcommand{\wh}{\widehat}
\newcommand{\wt}{\widetilde}
\newcommand{\thalf}{\textstyle \frac{1}{2}}
\newcommand{\nchoosek}[2]{\left(\begin{array}{c}#1\\#2\end{array}\right)}
\newcommand{\gap}{\vspace{0.1in}}
\newcommand{\epc}{\hspace{1pc}}
\newcommand{\sol}{\mbox{SOL}}
\newcommand{\lcp}{\mbox{LCP}}
\newcommand{\diag}{\mbox{Diag}}
\newtheorem{theorem}{Theorem}
\newtheorem{lemma}[theorem]{Lemma}
\newtheorem{lem}[theorem]{Lemma}
\newtheorem{proposition}[theorem]{Proposition}
\newtheorem{prop}[theorem]{Proposition}
\newtheorem{corollary}[theorem]{Corollary}
\newtheorem{cor}[theorem]{Corollary}
\newtheorem{remark}{Remark}
\newtheorem{rem}[theorem]{Remark}
\newtheorem{example}{Example}
\newtheorem{ex}{Example}
\newtheorem{definition}{Definition}
\newtheorem{defn}{Definition}
\newtheorem{thm}{Theorem}
\newtheorem{assumption}{Assumption}
\newcommand{\argmax}{\operatornamewithlimits{\arg\max}}
\newcommand{\argmin}{\operatornamewithlimits{\arg\min}}
\hyphenation{stoch-astic}

\begin{document}


%\title{Itinerary generation using a $k-$shortest paths approach}
\title{Robust shortest paths on directed acyclic graphs}

\author[1]{Udatta Palekar}
%\author[2]{Laxmikant Kale}
%\author[2]{Michael Robson}
%\author[2]{Akhil Langer}
\author[3]{Alok Tiwari\thanks{tiwari2@illinois.edu}}
\affil[1]{Department of Business Administration, University of Illinois at Urbana-Champaign}
\affil[2]{Department of Computer Science, University of Illinois at Urbana-Champaign}
\affil[3]{Department of Industrial \& Enterprise Systems Engineering, University of Illinois at Urbana-Champaign}


\renewcommand\Authands{, and }

%\author{University of Illinois at Urbana-Champaign, Urbana, Illinois 61801, U.S.A.}

\maketitle
\begin{abstract}
We describe a rational, systematic, and efficient approach to itinerary generation for dynamic mission replanning (DMR) problems using $k-$shortest paths. 
\end{abstract}

\section{Introduction}\label{sec:intro}
Consider the dynamic mission replanning (DMR) problems \cite{dmr-planning,dmr-execution}. The various variants of these problems consist of a large-scale stage-1 problem which may be viewed as a linear program or as an integer program and several large-scale stage-2 problems (at least one for each scenario) which too may be viewed as linear programs or as integer programs. The key variables in each subproblem are the \textit{itinerary} variables ($x_i$ or $x_{ii'}$ etc) which correspond to different itineraries. An itinerary itself is a path in (discretised) space-time that satisfies several `business rules' (such as respecting crew and aircraft `duty-cycles'). However, these business rules are often difficult to specify in advance, subject to change, or too complicated to incorporate directly into an optimizer. Therefore they are \textit{not} included in the DMR model constraints. In other words, the DMR models work with itineraries that have been filtered to satisfy business rules.

The present article proposes a unified, rational, algorithmic approach for generating itineraries suitable for the various DMR models. We noted above that an itinerary may be viewed as a path in discretised space-time. To incorporate duty-cycle rules, we just add another `dimension' to this (discrete) space-time and construct directed graphs on this set of grid points. The itineraries are then just directed paths in such graphs.

The rest of this article is organised as follows $\dots$ TBD.
\section{Robust shortest paths on a DAG}
%\documentclass{article}
%\usepackage{amsthm}
%\usepackage{verbatim}
%\newtheorem{thm}{Theorem}
%\begin{document}
%\long\def\/*#1*/{} %include multiline comments using \/* and */

%algorithm for the robust deviation shortest path
\paragraph{Algorithm 1 (robust deviation shortest path)}

Consider a (connected) directed acyclic graph $G$ with the vertex set $V = \{1, \dots, n\}$ and the edge set $E \subseteq V \times V$. We assume that the vertices are presented in a valid topological order. We also assume that there are a total of $m$ edges and $s$ scenarios (each scenario consists of a an assignment of weights or costs to edges). The weight of an edge $(i \rightarrow j)$ in scenario $s$ is denoted $w^s_{ij}$. Our goal is to find a path from $1$ to $n$ that minimizes the maximum deviation in cost across all scenarios.\cite{some_refs}

We now describe our algorithm:

\begin{description}
\item[Step 0] Solve the (ordinary) shortest path problems for each of the scenarios to obtain the quantities $l^s_j$, the cost of the \emph{shortest} path from $1$ to $j$ in scenario $s$.
\item[Step 1] For vertex $1$, define the quantities $c^s_1 = 0$, $\mu^s_1 = c^s_1 - l^s_1 = 0$, $\lambda_1 = \textrm{max}_s\left\{\mu^s_1\right\} = 0$. Here the superscript $s$ referes to the scenario and the subscript $1$ refers to the vertex. The quantities $c, ~\mu$ are respectively the \textit{cost} of the \emph{robust deviation} path to $1$,  and the cost deviation in choosing the robust path over the shortest path (these quantities depend on the scenario $s$). The quantity $\lambda$ is the \emph{maximum deviation} across all scenarios. 
\item[Step 2] For every vertex $j \in \left\{2,\dots, n\right\}$ let $P_j$ be the list of its immediate predecessors (presented in a topological order). We select a \emph{robust predecessor} to $j$ as follows:
	\begin{itemize}
		\item For $i \in R_j$, compute $\nu^s_i = c^s_i + w^s_{ij} - l^s_j$ and $\nu^{\textrm{max}}_i = \textrm{max}_s\left\{\nu^s_i\right\}$.
		\item Find $i_* = \textrm{argmax}_i\left\{\nu^{\textrm{max}_i}\right\}$.
	\end{itemize}
	Then $i_*$is the \emph{robust predecessor} of $j$.
\item[Step 3] Compute the quantities $c^s_j = c^s_{i_*} + w^s_{i_*j}$, $\mu^s_j = c^s_j - l^s_j$, and $\lambda_j = \textrm{max}_s\left\{\mu^s_j\right\}$.
\end{description}

\begin{thm}
The above algorithm produces a path that minimizes the maximum of the (cost) deviations from the shortest path in any scenario.
\end{thm}

\begin{proof}
We prove this result by induction on the vertices $j$ (which are considered in topological order). Clearly the basis of induction, i.e., $j=1$ is true: the trivial path from the vertex $1$ to itself is also the optimal robust deviation path. Now, as the induction hypothesis, let us assume that we have available the robust shortest paths from the vertex $1$ to any vertex $i$ where $i \le k$. We wish to show that, given this assumption, the above procedure produces a robust shortest path to the vertex $k+1$. This follows from the following lemma.
\end{proof}

\begin{lemma} If $(1,i_2,\dots,i_{k-1},i_k)$ is the sequence of vertices in a robust deviation shortest path from the vertex $1$ to the vertex $i_k$, then the truncated sequence $(1,i_1,\dots,i_{k-1})$ corresponds to a robust deviation shortest path from the vertex $1$ to the vertex $i_{k-1}$.
\end{lemma}
\begin{proof} We shall prove this result by contradiction. In other words, let us assume that there exists a path $\mathcal{P}_1$ from the vertex $1$ to the vertex $i_{k-1}$ that has a strictly smaller maximum deviation than the path $\mathcal{P}_0$ which corresponds to the above (truncated) sequence. Let $\mathcal{Q}_r$ be the continuation of the path $\mathcal{P}_r$ obtained by concatenating the edge $(i_{k-1} \rightarrow i_k)$ to it. Our assumptions above mean that $\mathcal{P}_1$ is the robust deviation shortest path from vertex $1$ to vertex $i_{k-1}$ whereas $\mathcal{Q}_0$ is the robust deviation shortest path from vertex $1$ to vertex $i_{k}$.

Let $\prescript{r}{}{c}^{s}_{i_{k-1}}$ be the cost of the path $\mathcal{P}_r$ in scenario $s$ and let ${l}^{s}_{i_{k-1}}$ be the cost of the shortest path from $1$ to $i_{k-1}$ in scenario $s$. Then, in scenario $s$, deviation of the (cost of the) path $\mathcal{P}_r$, is $\prescript{r}{}{\mu}^{s}_{i_{k-1}} = \prescript{r}{}{c}^{s}_{i_{k-1}} - {l}^{s}_{i_{k-1}}$. The optimality of $\mathcal{P}_1$ means that for some scenario $\tilde{s}$, we have:

\begin{equation*}
\prescript{1}{}{\mu}^{s}_{i_{k-1}} \le \prescript{1}{}{\lambda}_{i_{k-1}} < \prescript{0}{}{\lambda}_{i_{k-1}} = \prescript{0}{}{\mu}^{\tilde{s}}_{i_{k-1}} = \prescript{0}{}{c}^{\tilde{s}}_{i_{k-1}} - {l}^{\tilde{s}}_{i_{k-1}}
\end{equation*}

Note that the deviation from optimal for the path $\mathcal{Q}_r$ in scenario $s$ is

\begin{align*}
\prescript{r}{}{\mu}^{s}_{i_{k}} &= \left[\left({l}^{s}_{i_{k-1}} + \prescript{r}{}{\mu}^{s}_{i_{k-1}}\right) + w^s_{i_{k-1}i_{k}}\right] - {l}^{s}_{i_{k}}.\\
 &= \prescript{r}{}{\mu}^{s}_{i_{k-1}} + \left[\left({l}^{s}_{i_{k-1}} + w^s_{i_{k-1}i_{k}}\right) - {l}^{s}_{i_{k}}\right].
\end{align*}

\begin{assumption}
If we assume that the quantity $\left[\left({l}^{s}_{i_{k-1}} + w^s_{i_{k-1}i_{k}}\right) - {l}^{s}_{i_{k}}\right]$, which may be termed `edge deviation', is independent of the scenario $s$ (but may depend on the pair of vertices $i_{k-1}$ and $i_k$), we can complete the proof of the above lemma.
\end{assumption}

Under the above assumption, we note that there exists $\tilde{s}$ such that for all $s$, we have

\begin{align*}
\prescript{1}{}{\mu}^{s}_{i_{k}} &= \prescript{1}{}{\mu}^{s}_{i_{k-1}} + \left[\left({l}^{}_{i_{k-1}} + w^{}_{i_{k-1}i_{k}}\right) - {l}^{}_{i_{k}}\right]\\
&\le \prescript{1}{}{\lambda}^{}_{i_{k-1}}+ \left[\left({l}^{}_{i_{k-1}} + w^{}_{i_{k-1}i_{k}}\right) - {l}^{}_{i_{k}}\right]\\
&< \prescript{0}{}{\lambda}^{}_{i_{k-1}}+ \left[\left({l}^{}_{i_{k-1}} + w^{}_{i_{k-1}i_{k}}\right) - {l}^{}_{i_{k}}\right]\\
&= \prescript{0}{}{\mu}^{\tilde{s}}_{i_{k-1}}+ \left[\left({l}^{}_{i_{k-1}} + w^{}_{i_{k-1}i_{k}}\right) - {l}^{}_{i_{k}}\right]\\
&=\prescript{0}{}{\mu}^{\tilde{s}}_{i_{k}}
\end{align*}

Which contradicts the assumption that $\mathcal{Q}_0$ is the robust deviation shortest path from vertex $1$ to vertex $i_{k}$.

This completes the proof of the lemma.

\end{proof}



\begin{comment}
Therefore 

\begin{equation*}
\prescript{1}{}{\mu}^{s}_{i_{k}} \le \left[\left({l}^{s}_{i_{k-1}} + \prescript{1}{}\lambda_{i_{k-1}}\right) + w^s_{i_{k-1}i_{k}}\right] - {l}^{s}_{i_{k}} < \left[\left({l}^{s}_{i_{k-1}} + \prescript{0}{}\lambda_{i_{k-1}}\right) + w^s_{i_{k-1}i_{k}}\right] - {l}^{s}_{i_{k}}.
\end{equation*}
\end{comment}


\begin{comment}
Similarly, the optimality of $\mathcal{Q}_0$ means that 

\begin{equation*}
\prescript{0}{}{\mu}^{s}_{i_{k}} \le \prescript{0}{}{\lambda}_{i_{k}} \le \prescript{1}{}{\lambda}_{i_{k}} = \prescript{1}{}{\mu}^{\hat{s}}_{i_{k}} = \prescript{1}{}{c}^{\hat{s}}_{i_{k}} - {l}^{\hat{s}}_{i_{k-1}}
\end{equation*}
\end{comment}

\begin{comment}
First some notation: let the path produced by our algorithm be $\mathcal{P}^{\textrm{alg}} = (1 = v_1,\dots,v_k = n)$. Let $\mathcal{P} = (1 = w_1,\dots, w_l = n)$ be any other path. We shall show that the maximum deviation for the path $\mathcal{P}$ is no better than the maximum deviation for the path $\mathcal{P^{\textrm{alg}}}$. Suppose this is not so. Let $s_{\mathcal{P}}$ denote the scenario that maximises the deviation from optimal for the path $\mathcal{P}$. Furthermore let $r$ denote the minimal index such that $w_r \neq v_r$, and let $s$ denote the minimal index such that $w_{l-s} \neq v_{k-s}$. Clearly, $r \geq 2$ and $s \geq 1$.

We consider the path $\mathcal{P}^{\textrm{alg}}$ for the scenario $s_{\mathcal{P}}$. Our assumption that $\mathcal{P}$ has a smaller maximum deviation than $\mathcal{P^{\textrm{alg}}}$ implies that the cost of $\mathcal{P^{\textrm{alg}}}$ in the scenario $s_{\mathcal{P}}$ must be greater than the cost of $\mathcal{P}$. This implies that the subpath $(w_{r-1},\dots,w_{l-s+1})$ has a strictly lower cost than the subpath $(v_{r-1},\dots,v_{k-s+1})$ in the scenario $s_{\mathcal{P}}$ (these two subpaths share the start and end vertices). Without loss of generality\footnote{Requires a justification; not hard to write but TBD.}, we may assume that $v_f \neq w_g$ for all $r \le f \le k-s$ and $r \le g \le l-s$ -- so that the two subpaths do not share any vertices that aren't the start or the end. Consider the vertex $v_r$: the cost to reach it in the scenario $s_{\mathcal{P}}$
%\end{proof}
\end{comment}


\paragraph{Algorithm 2 ($k-$shortest robust deviation paths)}
\begin{description}
\item[Step 0:] For each scenario we construct a singe-source shortest path tree on the graph. This has a complexity of $\mathcal{O}(s m)$. At the end of this step we have an $(s+3)-$tuple as a label at each vertex consisting of its distances from the origin (vertex $1$) in each scenario, with two further labels, say, $-1$ and $+\infty$ added on to indicate the predecessor in the robust shortest path, and distance from the origin along the robust shortest path. At a vertex $i$ we denote its label as $(d_i^1,\dots,d_i^s,P_i,d_i^{\textrm{robust}},s^{*})$, where $s_i^{*}$ is the scenario for which the deviation is the greatest.

\item[Step 1:] For the source vertex define the robust path and calculate the cost of the robust path in each scenario\footnote{Using either the ARSP or MDSP definition. This distinction doesn't matter for the source vertex since all costs are zero.}.

\item[Step 2:] Process the remaining vertices in a valid topological order. For each vertex $v$, we determine a `robust path'\footnote{Or $k$-shortest paths} to it from the source as follows:
\begin{enumerate}
\item Let $\{e^v_1,\dots,e^v_l\}$ be all the incoming edges for the vertex $v$. Let us denote the corresponding predecessor vertices $\{u_1,\dots,u_l\}$ (we drop the dependence on $v$ from the notation). For each predecessor vertex $u_i$ we compute the following $S$ quantities, where $S$ is the number of scenarios: \begin{comment}$c^{\textrm{sp},s}_i + c^s_{u_iv}$ and\end{comment} 
$c^{\textrm{rp},s}_i + c^s_{u_iv}$. In the previous expression the superscript $\textrm{rp}$ indicates that the cost is for the `robust' path. The superscript $s$ indicates the scenario.
\item 

\end{enumerate}
\end{description}
\begin{comment}
\begin{description}
	\item[Step 0:] For each scenario we construct a singe-source shortest path tree on the graph. This has a complexity of $\mathcal{O}(s m)$. At the end of this step we have an $(s+3)-$tuple as a label at each vertex consisting of its distances from the origin (vertex $1$) in each scenario, with two further labels, say, $-1$ and $+\infty$ added on to indicate the predecessor in the robust shortest path, and distance from the origin along the robust shortest path. At a vertex $i$ we denote its label as $(d_i^1,\dots,d_i^s,P_i,d_i^{\textrm{robust}},s^{*})$, where $s_i^{*}$ is the scenario for which the deviation is the greatest.
	\item[Step 1:] We next go through the vertices in a topological order and, for each vertex $j$, we look at all its incoming edges $(i,j)$\footnote{It isn't necessary for the correctness of our algorithm but we go through the list of predecessors in the topological order as well.} and we update the last two entries of $j$'s label as follows:
		\begin{description}
			\item[Update $j$:] Let $d_{ij}^s$ be the weight on the edge $(i,j)$ in scenario $s$. We compute the biggest of the following quantities: $\{d_{ij}^s + d_i^s - d_j^s, d_{ij^{s_i^{*}}} + d_i^{\textrm{robust}} - d_j^{s_i^{*}}\}$, denoted $D_j^{\textrm{max}}$ which corresponds to scenario $s'$, and compare it to the current value of $d_j^{\textrm{robust}} - d_j^{s_j^{*}}$. If $D_j^{\textrm{max}}$ is smaller, the value $d_j^{\textrm{robust}}$ is replaced by $d_j^{s'} + D_j^{\textrm{max}}$ and the $P$ and $s$ values are updated correspondingly.
		\end{description}
		The complexity of this step is also $\mathcal{O}(s m)$.
	\item[Step 2:] Finally the path that minimizes the maximum deviation is produced by starting at vertex $n$ and looking at predecessors $P$ until we reach vertex $1$. The cost of this path in any scenario (and indeed the deviation from the minimum cost to reach vertex $n$ in any scenario) is trivial to calculate. The complexity of this step is no more than $\mathcal{O}(s m)$.
\end{description}
\end{comment}


%\end{document}
\section{Itinerary generation problem}\label{sec:problem}
\subsection{Problem description}\label{subsec:desc}
Consider the stage-1 `planning' problem, considered as a linear program. The notation below is the same as before. Additionally, we define the duals associated with each constraint:

\begin{longtable}[h]{lll}
{\it Objective Function} &&\\
minimize$\displaystyle\ \sum_{i} \zeta_i x_i$ + $\displaystyle\sum_{i} \displaystyle\sum_{lt} \Gamma^f_l F_{ilt} x_i$ $\displaystyle +~\sum_{s=0}^S p_s Q_s$& & \\
\ \ where: $\displaystyle Q_s := \sum_{e_s=1}^{N_s}\left(\sum_{i} \zeta_{i}^{e_s} x^{e_s *}_{i}~+~\sum_{i} \sum_{lt} \Gamma^f_l F_{ilt} x^{e_s *}_{i} + 
\sum_{f} P_f \mu_f^* + \sum_{i} cost_{fi} z_{fi}^{e_s *}\right)$, defined for scenarios $s$& &\\
\ \ which occur with probabilities $p_s$ and have $N_s$ events $e_s$.& &\\
%& & \\
\end{longtable}
\begin{longtable}[h]{ll|l|l}
%& & \\
{\it Constraint} && {\it Dual} &\\
{\it Aircraft Constraints} &&&\\
$\displaystyle\sum_{i=1}^{I} \phi_{ijlt}x_i - \displaystyle\sum_{i=1}^{I} \psi_{ijl(t-R_{j}')}x_i + a_{jlt} - a_{jl(t-1)}$ & $= N_{jlt}$ &$\pi^A_{jlt}$ & $\forall j,l,t$\\
& & &\\
{\it Crew Constraints} &&& \\
$\displaystyle\sum_{i=1}^{I} \eta_{iklt}x_i - \displaystyle\sum_{i=1}^{I} \gamma_{ijl(t-R_k)}x_i + c_{klt} - c_{kl(t-1)}$ & $= C_{klt}$ &$\pi^C_{klt}$& $\forall k,l,t$ \\ 
& & &\\
{\it Fuel Constraints} &&& \\
$\displaystyle\sum_{i=1}^{I} F_{ilt}x_i + f_{lt} -f_{l(t-1)} $ & $ = G_{lt}$ &$\pi^F_{lt}$ & $\forall l,t$\\
%$\displaystyle\sum_{i=1}^{I} F_{ilt}x_i + f_{lt} -f_{l(t-1)} + \sum_{i, h \in rhop(i)} F_{ih}'rdep(l,i,h,t)v_{ihl}$ & $ = G_{lt}$ &$\pi^F_{lt}$ & $\forall l,t$\\
& & &\\
{\it Maximum on Ground} &&& \\
$\displaystyle\sum_{i=1}^{I}gs(i)\theta_{ilt}x_i - \displaystyle\sum_{i=1}^{I}gs(i)\omega_{ilt}x_i + m_{lt} - m_{l(t-1)}$ & $=M_{lt}$ &$\pi^M_{lt}$ & $\forall l,t$ %\\ 

%$\displaystyle\sum_{i,j}gs(i)J^{L}_{j}J_{ij}\theta_{ilt}x_i - \displaystyle\sum_{i,j}gs(i)J^{L}_{j}J_{ij}\omega_{ilt}x_i + m'_{lt} - m'_{l(t-1)}$ & $=M^L_{lt}$ &$\pi^{M,{\cal SB}}_{jlt}$ & $\forall l \in {\cal SB},t$ \\ 
%{\it Refueling constraints} &&& \\
%$\displaystyle\sum_{l \in rloc(i,h)} v_{ihl} - \nu_{ih}x_i $ & $ =0 $ &$\pi^R_{ih}$ & $\forall i, h \in rhop(i)$ \\ 
%$\displaystyle\sum_{i,h \in rhop(i)} rdep(l,i,h,t)v_{ihl} - \displaystyle\sum_{i,h \in rhop(i)} rarr(l,i,h,t)v_{ihl} + t_{lt} - t_{l(t-1)}$ & $ =N_{jlt} $ &$\pi^T_{jlt}$ & $\forall l,t, j = $ tanker-type\\ 
%& & \\
%{\it Alternative for Refueling constraints} && \\
%$\displaystyle\sum_{i=1}^{I} \tau_{ilt}x_i - \displaystyle\sum_{i=1}^{I} \rho_{ilt}x_i + t_{lt} - t_{l(t-1)}$ & $= N_{jlt}$ & $\forall l,t, j = $tanker-type\\
\end{longtable}

\subsection{Input data}\label{subsec:data}
We are given a \underline{basic} feasible solution of the above LP, a corresponding `basis', and the dual values associated with this solution. We write the constraints above as $Ey = r$ where 

\begin{equation*}
y=\Big((x_i),(a_{jlt}),(c_{klt}),(f_{lt}),(m_{lt})\Big)%,(m_{lt}^{'}),(t_{lt})\Big)
\end{equation*}

and 

\begin{equation*}
E = \Big[[E^{x}_{i}],[E^{a}_{jlt}],[E^{c}_{klt}],[E^{f}_{lt}],[E^{m}_{lt}]\Big]%,[E^{'m}_{lt}],[E^{t}_{lt}]\Big].
\end{equation*}

Let $y$ be a basic feasible solution with the basis $\mathcal{B} = \mathcal{B}_x \cup \mathcal{B}_a \cup \mathcal{B}_c \cup \mathcal{B}_f \cup \mathcal{B}_m$. %\cup \mathcal{B}_{m'} \cup \mathcal{B}_t$.

Then the basis matrix is:
\begin{align*}
E_{\mathcal{B}} = & \Big[E_{\alpha}\Big]_{\alpha \in \mathcal{B}}\\
				= & \Big[[E^{x}_{i}]_{i \in \mathcal{B}_x},[E^{a}_{jlt}]_{{(j,l,t)} \in \mathcal{B}_a},[E^{c}_{klt}]_{{(k,l,t)} \in \mathcal{B}_c},[E^{f}_{lt}]_{{(l,t)} \in \mathcal{B}_f},[E^{m}_{lt}]_{{(l,t)} \in \mathcal{B}_m}\Big]%,[E^{'m}_{lt}]_{{(l,t)} \in \mathcal{B}_{m'}},[E^{t}_{lt}]_{(l,t) \in \mathcal{B}_t}\Big].
\end{align*}

And the reduced-cost for a non-basic column $E^x_i$ is 
\begin{align*}
& \left(\zeta_i + \sum_{lt} \Gamma^f_l F_{ilt}\right) ~-~ & \left[\left(\zeta_{i'} + \sum_{lt} \Gamma^f_l F_{i'lt}\right)_{i' \in \mathcal{B}_x}, 0, \dots, 0\right] E_{\mathcal{B}}^{-1} E^{x}_{i}\\
= & \left(\zeta_i + \sum_{lt} \Gamma^f_l F_{ilt}\right) ~-~ & \pi \cdot E^{x}_{i}\\
= & \left(\zeta_i + \sum_{lt} \Gamma^f_l F_{ilt}\right) ~-~ & \left(\sum_{j,l,t} (\phi_{ijlt} - \psi_{ijl(t-R'_j)})\pi^{A}_{jlt} + \sum_{k,l,t} (\eta_{iklt} - \gamma_{ikl(t-R_k)})\pi^{C}_{klt}\right.\\
 & & \ \ + \left.\sum_{l,t} F_{ilt}\pi^{F}_{lt} + \textrm{gs}(i) \sum_{l,t} (\theta_{ilt} - \omega_{ilt})\pi^{M}_{lt}\right)\\%\\
% & & \ \ + \sum \pi^{M,{\cal SB}}_{lt} + \sum_{h \in \hops(i)}\nu_{ih}\pi^{R}_{ih} + \sum \pi^{T}_{jlt}
= & \left(\zeta_i + \sum_{lt} \Gamma^f_l F_{ilt}\right) ~-~ & \left(\sum_{j,l,t} \phi_{ijlt}\pi^{A}_{jlt} - \sum_{j,l,t}\psi_{ijlt}\pi^{A}_{jl(t+R'_j)} + \sum_{k,l,t} \eta_{iklt}\pi^{C}_{klt} - \sum_{k,l,t}\gamma_{iklt}\pi^{C}_{kl(t+R_k)}\right.\\
 & & \ \ + \left.\sum_{l,t} F_{ilt}\pi^{F}_{lt} + \textrm{gs}(i) \sum_{l,t} (\theta_{ilt} - \omega_{ilt})\pi^{M}_{lt}\right)%\\
\end{align*}

%We denote this basic feasible solution by $\Big((x_i^{b}),(a_{jlt}^{b}),(c_{klt}^{b}),(f_{lt}^{b}),(m_{lt}^{b}),(m_{lt}^{'b}),(t_{lt}^{b})\Big)$. 
\section{Graph description}\label{sec:graph}
\subsection{Nodes}\label{subsec:nodes}
The nodes in our graph consist of triples of the form $(l,t,d)$ where $l \in \mathcal{L}$, the set of locations\footnote{Each geographical location gets represented \underline{twice} in the set $\mathcal{L}$, once as an arrival location $l_A$ and once as a departure location $l_D$.}, $t \in \left\{ 0,\dots,\mathcal{N}_{\textrm{periods}}\right\}$, and $d \in \left\{0,\dots,\mathcal{M}_{\textrm{max duty periods}}\right\}$.
\subsection{Edges}\label{subsec:edges}
There are several different classes of edges in this graph. We enumerate these classes below:

\begin{enumerate}
	\item One edge that goes from an arrival node $(l_A,t,d)$ to a departure node $(l_D,t+\Delta,\max\{d+\Delta,\mathcal{M}\})$.
		\begin{itemize}
			\item Depending on the nature of the stop, the aircraft needs to remain on ground for a minimum length of time; this determines the value of $\Delta$.
		\end{itemize}
	\item Edges that go from a departure node $(l_D,t,d)$ to an arrival node $(l'_A,t+TT,\max\{d+TT,\mathcal{M}\})$\footnote{For some departure nodes, e.g., those with `high' values of $d$, there will be no such edge since there aren't enough crew duty hours available to complete the next leg.}.
		\begin{itemize}
			\item These edges represent the travel between two locations. $TT$ is the travel time between the locations.
			\item Special case: we allow for 'in-air delays' of up to $D_\textrm{max delay}$. In other words $TT$ can take the values $\{TT_{\textrm{min}},TT_{\textrm{min}}+1,\dots,TT_{\textrm{min}}+D_\textrm{max delay}\}$.
		\end{itemize}
	%\item Edges that go from a departure node to another departure node. These include:
	%	\begin{itemize}
	%		\item Ground delays: edges that go from $(l_D,t,d)$ to $(l_D,t+1,\max\{d+1,\mathcal{M}\})$.
	%		\item Rest overnight: edges that go from $(l_D,t,d)$ to $(l_D,t+\mathcal{T}_{\textrm{RON}},0)$.
	%		\item Crew swap: edges that go from $(l_D,t,d)$ to $(l_D,t+\mathcal{T}_{\textrm{CS}},0)$.
	%		\item Special case: The only out-going edge from a node of the form $(l_D,t,\mathcal{M})$ is to the node $(l_D,t+\mathcal{T}_{\textrm{RON}},0)$
	%	\end{itemize}
\end{enumerate}
\subsection{Node and edge weights}\label{subsec:weights}
The edge weights are determined as follows (note that the parameters $i,j,k$ are known):
\begin{itemize}
\item An edge that goes from an `arrival' node $({l_{0}}_A,t_0,d_0)$ to a `departure' node $({l_{0}}_D,t_1,d_1)$ carries a weight $\displaystyle \eta_{ijlt_1}\pi^{A}_{l_0t_1} - \psi_{ijlt_0}\pi^{A}_{l_0(t_0+R'_j)} + \eta_{iklt_1}\pi^{C}_{l_0t_1} - \gamma_{iklt_0}\pi^{C}_{l_0(t_0+R_k)} + F_{il_0t_1}\pi^{F}_{l_0t_1} + \textrm{gs}(i)\sum_{t_0 \le t \le t_1} \pi^{M}_{l_0t}$%\left(\theta_{il_0t_1}\pi^{M}_{l_0t_1} - \omega_{il_0t_0}\pi^{M}_{l_0t_0}\right) + F_{il_0t_1}\pi^{F}_{l_0t_1}$.%\sum_{t_0 \le t \le t_1} \textrm{gs}(i)\pi^{M}_{l_0t}$
\item An edge that goes from a `departure' node $({l_0}_D,t_0,d_0)$ to an `arrival' node $({l_1}_D,t_1,d_1)$ carries a weight $\Gamma^{f}_{l_0}F_{il_0t_0}\pi^{F}_{l_0t_0}$.
\item We don't yet know where to assign $\zeta_i$ as an edge-weight. 
\end{itemize}
%\[
%\textrm{Edge weight} ~=~ \sum_{t \ge (t_1-t_c)}\pi^{\textrm{crew}}_{kl_1t}\eta_{ikl_1t_1} + \sum_{t \ge (t_1-t_a)}\pi^{\textrm{acft}}_{jl_1t}\phi_{ijl_1t_1} + \pi^{\textrm{MoG}}_{l_1t_1}\theta_{il_1t_1}\textrm{gs}_j 
%- \sum_{t \ge (t_0+R_k)}\pi^{\textrm{crew}}_{kl_0t}\gamma_{ikl_0t_0} - \sum_{t \ge (t_0+R'_j)}\pi^{\textrm{acft}}_{jl_0t}\psi_{ijl_0t_0}
%\]
\newpage

where:
\begin{itemize}
\item The edge goes from a node of the form $(l_0,t_0,d_0)$ to a node of the form $(l_1,t_1,d_1)$.
\item $t_c$ is the time before take-off that a crew is ``picked up''.
\item $t_a$ is the time before take-off that an aircraft is picked up.
\item $R_k$ is the recuperation time (assumed to be fixed) for a crew that is dropped off.
\item $R'_j$ is the recuperation time (assumed to be fixed) for an aircraft that is dropped off.
\end{itemize}

%\begin{enumerate}
%\item Each `travel' edge carries a weight equal to the flying cost.
%\item Each edge that begins and ends at the same location carries zero weight.\footnote{This can be changed, depending on modeling choices. For example, an edge corresponding to a crew swap may carry a weight equal to crew costs.}
%\item Each node carries a weight obtained from the present values of the `duals' (e.g., an $(l,t)$ pair that is MoG constrained may carry a higher weight than one that has plenty of spare capacity).
%\end{enumerate}

\subsection{Weather disruptions}\label{subsec:weather}
If a weather disruption prevents landings or take-offs from certain $(l,t)$ pairs, we simply remove appropriate edges from the graphs. If this disruption limits the number of landings or take-offs, we may be able to discourage landings or take-offs by increasing the weights of such edges.
\section{Algorithm}\label{sec:algorithm}
With the above (directed) graph, we simply solve a standard $k-$shortest path problem where the `length' of a path is the sum of the weights of all the edges and nodes that it contains. This yields $k$ itineraries that satisfy crew rules.
\section{Numerical results}\label{sec:numerics}
\input{numerics}
\end{document}

